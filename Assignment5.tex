\documentclass[11pt]{article}

\usepackage{xcolor}
\usepackage{listings}
\usepackage{graphicx}
\usepackage{subfigure}
\lstset{ %
language=C++,                % choose the language of the code
basicstyle=\footnotesize,       % the size of the fonts that are used for the code
backgroundcolor=\color{white!3!white},  % choose the background color. You must add \usepackage{color}
showspaces=false,               % show spaces adding particular underscores
showstringspaces=false,         % underline spaces within strings
showtabs=false,                 % show tabs within strings adding particular underscores
frame=single,           % adds a frame around the code
tabsize=2,          % sets default tabsize to 2 spaces
captionpos=b,           % sets the caption-position to bottom
breaklines=true,        % sets automatic line breaking
breakatwhitespace=false,    % sets if automatic breaks should only happen at whitespace
escapeinside={\%*}{*)},          % if you want to add a comment within your code
  basicstyle=\footnotesize\ttfamily,
  keywordstyle=\bfseries\color{green!40!black},
  commentstyle=\itshape\color{gray!80!black},
  identifierstyle=\color{black},
}


\title{\textbf{Computer Graphics : Assignment 5}}

\author{
		\vspace{ 2 mm}\\
		Supervised By : \textbf{Prof. Srikanth TK}\\
		\vspace {2mm}\\
		amit.tomar@iiitb.org \\
		(MT2013008) \\
		}
		
\date{21 - April - 2014}

\begin{document}
\lstset{language=C} 
\maketitle

\vspace{ 100 mm}

\section{Prerequisite Libraries}

\begin {enumerate}
\item OpenGl (Mesa libraries)
\item Qt5 libraries
\end {enumerate}

\section{Build Steps}

\begin {enumerate}
\item Change directory to folder MT2013008\_Assignment5/Src
\item Run the following command\\

\textbf{qmake MT2013008\_Assignment5.pro -o ../Build/Makefile} \\

This will generate the Makefile in the Build folder.

\item Change directory to folder MT2013008\_Assignment5/Build

\item Run the following command \\

\textbf{make} \\

This will build the complete project and generate and executable in the Build directory.

\item To start the application, go to build directory, execute following command 

\textbf{./MT2013008\_assignment5}



\end {enumerate}

\section{Usage}

\textbf{Note}: To enable keyboard controls, left-click on the left side of the screen.

\begin {enumerate}
\item Application starts with a file browser.
\item Go to plyfiles folder and select a ply file to render.
\item Use 'p',';' keys to scale up or down.
\item Use 'a','s','d','w' keys to translated left, right, up, down.
\item Use 't','f','g','h' keys to rotate object left, right, up, down. Or use the sliders at right. Or left-click on the screen and rotate with mouse.
\item Use 'i','j','k','l' keys to rotate light left, right, up, down.
\item Use key 'q' to enable/disable the movement of light with object.
\item \textbf{Use key 'n' to sub divide}.
\item \textbf{Use key 'm' to navigate through the points of control mesh}.
\item \textbf{Use key 'z' to move the currently selected point of control mesh}.
\item \textbf{Use key 'e' to export the sub divided mesh as a ply file}.


\end {enumerate}

\section{ Issues with implementation}

\begin {enumerate}
\item Deciding the correct data structure is the most crucial aspect of this assignment. I did not fix upon all the use cases initially, and thus had to do lot of modifications later on.
\item Calculating the correct normal direction (Inwards/Outwards) (Still pending).
\item If the input data was a list of triples of vertices, some change in parser will be required.
\end {enumerate}             

\section{ Performance}

\begin {enumerate}
\item After three sub-divisions, it was taking particularly long time to sub-divide.
\item Process of editing the control mesh is not very responsive (No use of mouse).
\item For large size models, too much time was required for sub-division, even for a single sub-division.
\end {enumerate}   

\end{document}

